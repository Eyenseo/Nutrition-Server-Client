\documentclass[english,a4paper,12pt]{scrartcl}
\usepackage[utf8]{inputenc}
\usepackage{datetime}
\usepackage[bookmarks,colorlinks=true,urlcolor=blue]{hyperref}
\usepackage[automark,headsepline]{scrlayer-scrpage}
\usepackage[vscale=0.75,vmarginratio={2:3},heightrounded]{geometry} % less margin at bottom
\usepackage{minted}
\usepackage{enumitem} %for no space in itemnize
%---------------------------------
% Meta data
%---------------------------------
\hypersetup{pdfauthor={Roland Jäger},
            pdftitle={Distributed Communication},
            pdfsubject={Systems Programming, Assignment 2},
            pdfkeywords={Distributed Communication, Systems Programming,Assignment 2, Queensland University of Technology}}

%---------------------------------
% Fancy header
%---------------------------------
\clearpairofpagestyles
\lohead{\headmark}
\lohead{\headmark}
\cofoot[\pagemark]{\pagemark}
\pagestyle{scrheadings}

%---------------------------------
% Start page count
%---------------------------------
\setcounter{page}{0}

%---------------------------------
% Minted color scheme
%---------------------------------
\usemintedstyle{borland}

%---------------------------------
% Horizontal line for title page
%---------------------------------
\newcommand{\HRule}{\rule{\linewidth}{0.5mm}} % Defines a new command for the horizontal lines, change thickness here

%---------------------------------
% href Link as foot note
%---------------------------------
\newcommand\fnurl[2]{
  \href{#2}{#1}\footnote{\url{#2}}
}
%---------------------------------
% Due date for title page
%---------------------------------
\newdate{dueDate}{21}{10}{2014}

\begin{document}
\begin{titlepage}
  \center % Center everything on the page

  %---------------------------------
  % HEADING SECTIONS
  %---------------------------------
  ~\\[2cm]
  \textsc{\LARGE Queensland University of Technology}\\[1.5cm] % Name of your university/college
  \textsc{\Large Systems Programming}\\[0.5cm] % Major heading such as course name
  \textsc{\large Assignment 2}\\[2cm] % Minor heading such as course title

  %---------------------------------
  % TITLE SECTION
  %---------------------------------
  \HRule \\[1cm]
  {
    \huge \bfseries Distributed Communication
  }\\[0.5cm] % Title of your document
  \HRule \\[1.5cm]

  %---------------------------------
  % AUTHOR SECTION
  %---------------------------------
  \begin{minipage}{0.4\textwidth}
    \begin{flushleft} \large
      \emph{Author:}\\
      Roland  \textsc{Jäger} \\% Your name
      \small n9247220
    \end{flushleft}
  \end{minipage}
  ~
  \begin{minipage}{0.4\textwidth}
    \begin{flushright} \large
      \emph{Due Date:} \\
      \shortdate\displaydate{dueDate}\\~
    \end{flushright}
  \end{minipage}\\[4cm]

  %---------------------------------
  % DATE SECTION
  %---------------------------------
  \vspace*{\fill}
  {
    \large \today
  }
\end{titlepage}

\section{About the Programs}
  The programs full fill the tasks one, two and three without exceptions.

  \subsection{Deviations}
    \begin{itemize}[itemsep=5pt,topsep=0pt,parsep=0pt,partopsep=0pt]
      \item
        For the server the signal handling is not done using the function \linebreak \mintinline{c}{void (*signal(int sig, void (*func)(int)))(int);}  because of thread safety. Also the usage of a signal to handle termination is optional -- from the programs view. The accept loop is in a thread and with that a normal ui interaction would be possible.
      \item
        The server features a queue that will always accept new clients, the queue is fed to a thread pool that will then answer client requests -- clients will wait until they receive a start signal from the server.
      \item
        The server has two booleans with dataraces -- they are used to toggle the state of the server. THey where left in since it doesn't matter if a thread misses the `first' signal to shut down\footnote{threads only read or set the variable to false} and the mutexes to guard the variables were rated worse (performance).
      \item
        Both, the server and client have a default port and IP address to the programs can be started without arguments for convenience purpose.
    \end{itemize}

  \subsection{Client Arguments}
    \texttt{
      ./client [IP-ADDRESS] [PORT]
    }
    \begin{itemize}[itemsep=5pt,topsep=3pt,parsep=0pt,partopsep=0pt]
      \item[] The \texttt{IP-ADDRESS} Argument is optional (but is needed to specify \texttt{PORT}) and defaults to \texttt{127.0.0.1}
      \item[] The \texttt{PORT} Argument is optional and defaults to \texttt{12345}
    \end{itemize}
  \subsection{Server Arguments}
    \texttt{
      ./server [PORT]
    }
    \begin{itemize}[itemsep=5pt,topsep=3pt,parsep=0pt,partopsep=0pt]
      \item[] The \texttt{PORT} Argument is optional and defaults to \texttt{12345}
    \end{itemize}

\section{Remarks}
  There are similarities in my network code and the code one can find in \fnurl{Beej's Guide to Network Programming}{http://beej.us/guide/bgnet/output/html/singlepage/bgnet.html} as this was the main information resource regarding the network aspect of the assignment.

\end{document}
